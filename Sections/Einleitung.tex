Der folgende Abschnitt beschreibt die Motivation und die grundlegenden Ziele des Projekts.

\subsection{Motivation}
Als Motivation diente ein Sudoku-Heft mit bestimmten partiellen Sudokus,
welche ”Bilder” von Katzen darstellen.
Die Idee war, ein Programm zu entwickeln, mit welchem man solche Sudokus generieren kann.

\subsection{Ziele}
Das Ziel des Projekts ist die Bereitstellung eines grafischen Benutzer-Interfaces (GUI) für das Erstellen
von partiellen Sudoku’s mit vorgegebenen Zellen, welche ein bestimmtes Muster erzeugen.
Außerdem soll es möglich sein für einige dieser Zellen explizit Ziffern festzulegen.
Des weiteren soll für ein partielles Sudoku dessen Schwierigkeit abgeschätzt werden können.

\subsection{Aufbau}
In \cref{sec:theoretische_ueberlegungen} werden einige theoretischen Vorüberlegungen zu Sudokus angestellt.
Anschließend wird in \cref{sec:backend} das Backend beschrieben.
Das Backend ist zum einen für die Generierung von partiellen Sudokus zuständig,
zum anderen für das Lösen von Sudokus und für die Bewertung der Schwierigkeit.
Den Abschluss bilden die Abschnitte \cref{sec:frontend} und \cref{sec:kommunikation},
welche vom Frontend und der Kommunikation zwischen Frontend und Backend handeln.