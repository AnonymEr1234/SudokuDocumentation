\subsection{Symmetrien}
Ein Sudoku besteht aus 3 horizontalen und 3 vertikalen Bändern. Die Bänder bestehen jeweils aus 3 Zeilen beziehungsweise 3 Spalten. \\
Die Tupel (1, 2, 3),(4, 5, 6),(7, 8, 9) beschreiben die Zeilen oder Spalten der jeweiligen Bänder. \\
Mit dem Wissen über die Bänder können wir die Symmetrien eines Sudokus formulieren~\cite{russell2006mathematics}: \\
\begin{itemize}
    \item Permutation der 9 Ziffern
    \item Permutation der 3 horizontalen Bänder
    \item Permutation der 3 vertikalen Bänder
    \item Permutation der Zeilen innerhalb eines horizontalen Bandes
    \item Permutation der Spalten innerhalb eines vertikalen Bandes
    \item Spiegelung und Rotation
\end{itemize}
Insgesamt gibt es in etwa 6,7 Trilliarden verschiedene valide Sudokus \cite{felgenhauer2006mathematics}. Unter Berücksichtigung der Symmetrien
gibt es ungefähr 5,5 Milliarden verschiedene valide Sudokus\cite{russell2006mathematics}. \\
