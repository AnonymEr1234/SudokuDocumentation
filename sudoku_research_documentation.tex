
\documentclass[12pt,a4paper]{article}

% Packages
\usepackage[utf8]{inputenc}
\usepackage{amsmath}
\usepackage{amsfonts}
\usepackage{amssymb}
\usepackage{graphicx}
\usepackage{hyperref}
\usepackage{algorithm}
\usepackage{algpseudocode}
\usepackage{booktabs}
\usepackage{listings}

% Document settings
\title{Sudoku Forschungsprojekt Dokumentation}
\author{Bastian Fischer, Samuel Jaschke, Hannes Träger, Paul Volk}
\date{\today}

\begin{document}

\maketitle

\begin{abstract}
\end{abstract}

\section{Einführung}
\subsection{Hintergrund}

\subsection{Forschungsziele}
Es werden folgende Forschungsziele behandelt:
\begin{itemize}
    \item  
\end{itemize}

\section{Bisherige Literatur}
Für Sudokus mit 16 oder weniger Hinweisen existieren keine eindeutigen Lösungen \cite{DBLP:journals/corr/abs-1201-0749} somit muss ein Sudoku, damit es eine eindeutige Lösung hat mindestens 17 Hinweise haben.
\section{Methodik}

\subsection{Problem Formulierung}

\subsection{Symmetrien}
Ein Sudoku besteht aus 3 horizontalen und 3 vertikalen Bändern. Die Bänder bestehen jeweils aus 3 Zeilen beziehungsweise 3 Spalten. \\
Die Tupel (1, 2, 3),(4, 5, 6),(7, 8, 9) beschreiben die Zeilen oder Spalten der jeweiligen Bänder. \\
Mit dem Wissen über die Bänder können wir die Symmetrien eines Sudokus formulieren \cite{russell2006mathematics}: \\
\begin{itemize}
    \item Permutation der 9 Ziffern
    \item Permutation der 3 horizontalen Bänder
    \item Permutation der 3 vertikalen Bänder
    \item Permutation der Zeilen innerhalb eines horizontalen Bandes
    \item Permutation der Spalten innerhalb eines vertikalen Bandes
    \item Spiegelung und Rotation
\end{itemize}

\subsection{Algorithmen}
\subsubsection{Generierungsalgorithmen}
\begin{algorithm}
\caption{Sudoku generierung}
\begin{algorithmic}[1]
\State Ein vollständiges, gültiges Sudoku-Gitter generieren
\State Zellen systematisch entfernen und dabei eine eindeutige Lösung beibehalten
\State Überprüfen, ob der Schwierigkeitsgrad den Zielparametern entspricht
\end{algorithmic}
\end{algorithm}

\subsubsection{Lösungsalgorithmen}

\subsection{Implementierung}
\subsubsection{Tech Stack}
\subsubsection{Frontend}
\subsubsection{Backend}
\subsubsection{Interface}

\section{Ergebnisse}

\section{Diskussion}

\section{Fazit und Ausblick}

\bibliographystyle{plain}
\bibliography{references}

\end{document}
