
\documentclass[12pt,a4paper]{article}

% Packages
\usepackage[utf8]{inputenc}
\usepackage{amsmath}
\usepackage{amsfonts}
\usepackage{amssymb}
\usepackage{graphicx}
\usepackage{hyperref}
\usepackage{algorithm}
\usepackage{algpseudocode}
\usepackage{booktabs}
\usepackage{listings}

% Document settings
\title{Sudoku Forschungsprojekt Dokumentation}
\author{Bastian Fischer, Samuel Jaschke, Hannes Träger, Paul Volk}
\date{\today}

\begin{document}

\maketitle

\begin{abstract}
\end{abstract}

\section{Einführung}
\subsection{Hintergrund}

\subsection{Forschungsziele}
Es werden folgende Forschungsziele behandelt:
\begin{itemize}
    \item  
\end{itemize}

\section{Bisherige Literatur}
Für Sudokus mit 16 oder weniger Hinweisen existieren keine eindeutigen Lösungen \cite{DBLP:journals/corr/abs-1201-0749} somit muss ein Sudoku, damit es eine eindeutige Lösung hat mindestens 17 Hinweise haben.
\section{Methodik}

\subsection{Problem Formulierung}

\subsection{Symmetrien}
Ein Sudoku besteht aus 3 horizontalen und 3 vertikalen Bändern. Die Bänder bestehen jeweils aus 3 Zeilen beziehungsweise 3 Spalten. \\
Die Tupel (1, 2, 3),(4, 5, 6),(7, 8, 9) beschreiben die Zeilen oder Spalten der jeweiligen Bänder. \\
Mit dem Wissen über die Bänder können wir die Symmetrien eines Sudokus formulieren \cite{russell2006mathematics}: \\
\begin{itemize}
    \item Permutation der 9 Ziffern
    \item Permutation der 3 horizontalen Bänder
    \item Permutation der 3 vertikalen Bänder
    \item Permutation der Zeilen innerhalb eines horizontalen Bandes
    \item Permutation der Spalten innerhalb eines vertikalen Bandes
    \item Spiegelung und Rotation
\end{itemize}
Insgesamt gibt es in etwa 6,7 Trilliarden verschiedene valide Sudokus \cite{felgenhauer2006mathematics}. Unter Berücksichtigung der Symmetrien
gibt es ungefähr 5,5 Milliarden verschiedene valide Sudokus \cite{russell2006mathematics}. \\
Diese Symmetrien nutzen wir um die Anzahl an Belegungen des partiellen Sudokus zu reduzieren. Da beliebige Permutationen der 9 Ziffern  erlaubt sind,
können wir die erste Zeile aufsteigend belegen. Das heißt, dass wir die Ziffern von 1 bis 9 in der ersten Zeile anordnen. \\
Durch die Permutation der drei horizontalen Bänder und der Permutation der Zeilen innerhalb eines horizontalen Bandes, können wir außerdem bestimmte Regeln für die erste Spalte aufstellen:
\begin{itemize}
    \item $1< 2< 3 \text{ und } 4<5<6 \text{ und } 7<8<9$
    \item $4 < 7$
    \item Der erste Eintrag hat den Wert 1 (wegen der Belegung der ersten Zeile)
    \item Die Werte vom 2. und 3. Eintrag sind jeweils größer als 3 (wegen der Regel für die Blöcke und der Belegung der ersten Zeile).
\end{itemize}
Mit diesen Einschränkungen ergeben sich für ein 4x4 Sudoku 2 mögliche Belegungen der ersten Spalte. Für ein 6x6 Sudoku ergeben sich 9 mögliche
Belegungen der ersten Spalte und für ein 9x9 Sudoku ergeben sich 150 mögliche Belegungen der ersten Spalte. \\

\subsection{Algorithmen}
\subsubsection{Generierungsalgorithmen}
\begin{algorithm}
\caption{Sudoku Generierung}
\begin{algorithmic}[1]
\State Ein vollständiges, gültiges Sudoku-Gitter generieren
\State Zellen systematisch entfernen und dabei eine eindeutige Lösung beibehalten
\State Überprüfen, ob der Schwierigkeitsgrad den Zielparametern entspricht
\end{algorithmic}
\end{algorithm}

\subsubsection{Lösungsalgorithmen}

\subsection{Implementierung}
\subsubsection{Tech Stack}
\subsubsection{Frontend}
\subsubsection{Backend}
\subsubsection{Interface}

\section{Ergebnisse}

\section{Diskussion}

\section{Fazit und Ausblick}

\bibliographystyle{plain}
\bibliography{references}

\end{document}
