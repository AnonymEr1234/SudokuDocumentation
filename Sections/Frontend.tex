Das Frontend ist eine Webanwendung, die vollständig im Browser lauffähig ist.
Diese bietet zwei grundlegende Betriebsmodi für die Erstellung von Sudokus an.
Einmal den Markiermodus, in dem Felder visuell hervorgehoben werden können,
und den Eingabemodus, der eine manuelle Eingabe von Zahlen erlaubt.
Die Anwendung unterstützt dabei Sudoku-Größen von 4x4, 6x6 und 9x9.
\\
Die HTML-Datei enthält die grundlegenden Interface-Komponenten der Anwendung. Hierzu gehören:
\begin{itemize}
    \item Schaltflächen zur Auswahl der Sudoku-Größe (4x4, 6x6, 9x9), mit denen dynamisch ein passendes Raster erzeugt wird.
    \item Umschalter zwischen Markier- und Eingabemodus, die das Verhalten der Zellen beim Anklicken verändern.
    \item Das Grid welches beim anklicken verändert werden kann und wo dann bestimmte Muster oder Formen markiert werden können.
    \item Ein Download-Button, der es ermöglicht, das Sudoku als PDF-Datei zu exportieren.
\end{itemize}

Je nach gewählter Sudoku-Größe wird ein entsprechendes Raster dynamisch erzeugt.
Die Zellen erhalten je nach Modus unterschiedliche Eventlistener:
im Markiermodus lassen sie sich an- und abwählen, im Eingabemodus können Zahlen eingegeben werden.
Die Anwendung erlaubt es, zwischen einem Markier- und einem Eingabemodus zu wechseln.
Der aktuelle Modus wird zentral verwaltet, sodass alle Eingabefunktionen sowie Prüfmechanismen entsprechend angepasst sind.
Nach jeder Benutzereingabe wird automatisch überprüft, ob das aktuelle Sudoku gültig ist.
Im Eingabemodus wird dabei auf doppelte Zahlen in Zeilen, Spalten und Blöcken geachtet.
Im Markiermodus wird geprüft, ob genügend Felder markiert sind und ob die Verteilung über das Sudoku-Feld ausreichend ist (mindestens zwei Zeilen und Spalten pro Block müssen belegt sein).
Fehlerhafte Zellen werden visuell hervorgehoben. Die Benutzereingaben werden in ein JSON-Objekt umgewandelt und an eine serverseitige API übermittelt.
Diese verarbeitet das unvollständige Sudoku, berechnet eine mögliche Lösung und sendet sie zurück an den Client.
Die Lösung wird anschließend im dafür vorgesehenen Bereich angezeigt.
\\
Ein besonderes Augenmerk lag bei der Umsetzung auf der Validierung der Sudoku-Regeln.
Dabei musste berücksichtigt werden, dass bei Sudokus der Größen 4x4 und 6x6 die Blockgrößen variieren (2x2 bzw. 2x3).
Die Logik erkennt automatisch die korrekten Blockdimensionen und wendet die Regeln entsprechend an.
Zudem wurde eine eigene Fehlercodierung entwickelt,
um unterschiedliche Arten von Regelverstößen klar voneinander unterscheiden und gezielt anzeigen zu können.
Dazu gehören doppelte Werte, fehlende Verteilung sowie unvollständige Eingaben.
\\
Eine der größten Herausforderungen war die dynamische Erstellung und Validierung unterschiedlich großer Sudoku-Raster sowie die Anpassung der Prüf- und Eingabelogik an zwei Betriebsmodi.
Ebenso anspruchsvoll war die Entwicklung einer robusten Fehlererkennung,
die bei teilweise ausgefüllten Rätseln dennoch konsistente Rückmeldungen geben kann.