\documentclass{article}
\usepackage[utf8]{inputenc}
\usepackage[ngerman]{babel}

\title{Kommunikation zwischen Frontend und Backend}
\author{}
\date{}

\begin{document}

\maketitle

\section*{Kommunikation per POST-Anfrage}

Die Kommunikation zwischen Frontend und Backend erfolgt über POST-Anfragen. Das Frontend sendet dabei Daten im JSON-Format an einen definierten API-Endpunkt des Backends.

\section*{Frontend-Seite}

Das Frontend erstellt ein JSON-Objekt mit den benötigten Daten und schickt es per POST an das Backend. Beispiel mit JavaScript:

\begin{verbatim}
fetch('/api/endpoint', {
  method: 'POST',
  headers: { 'Content-Type': 'application/json' },
  body: JSON.stringify({ key: 'value' })
})
.then(response => response.json())
.then(data => {
  // Daten verarbeiten und UI aktualisieren
});
\end{verbatim}

\section*{Backend-Seite}

Das Backend empfängt die POST-Anfrage, liest die JSON-Daten aus, verarbeitet sie (z.B. Validierung, Datenbankzugriff) und sendet eine JSON-Antwort zurück.

\section*{Antwort}

Die Antwort enthält den Status der Verarbeitung und ggf. weitere Daten im JSON-Format. Das Frontend wertet diese aus und reagiert entsprechend (z.B. Anzeige von Erfolg oder Fehler).

\section*{Zusammenfassung}

Die Kommunikation basiert auf dem Senden und Empfangen von JSON-Daten per POST-Anfrage. Das Frontend baut die Anfrage, das Backend verarbeitet sie und antwortet mit JSON. So findet der Datenaustausch konkret statt.

\end{document}
