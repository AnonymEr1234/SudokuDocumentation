\documentclass[12pt,a4paper]{article}

% Packages
\usepackage[utf8]{inputenc}
\usepackage{amsmath}
\usepackage{amsfonts}
\usepackage{amssymb}
\usepackage{graphicx}
\usepackage{hyperref}
\usepackage{algorithm}
\usepackage{algpseudocode}
\usepackage{booktabs}
\usepackage{listings}
\usepackage{enumerate}
\usepackage{pythontex}
\usepackage{lstmisc}


\usepackage{tikz}
\usepackage{xcolor}
\usepackage{siunitx}
\usepackage{float}
\usepackage{cleveref}

% Document settings
\title{Sudoku Forschungsprojekt Dokumentation}
\author{Bastian Fischer, Samuel Jaschke, Hannes Träger, Paul Volk}
\date{\today}

\begin{document}

\maketitle

\begin{abstract}
\end{abstract}

\section{Einführung}
\subsection{Hintergrund}

\subsection{Forschungsziele}
Es werden folgende Forschungsziele behandelt:
\begin{itemize}
    \item  
\end{itemize}

\section{Bisherige Literatur}
Für Sudokus mit 16 oder weniger Hinweisen existieren keine eindeutigen Lösungen~\cite{DBLP:journals/corr/abs-1201-0749} somit muss ein Sudoku, damit es eine eindeutige Lösung hat mindestens 17 Hinweise haben.

\section{theoretische Überlegungen}

\subsection{Problem Formulierung}

\subsection{Symmetrien}
Ein Sudoku besteht aus 3 horizontalen und 3 vertikalen Bändern. Die Bänder bestehen jeweils aus 3 Zeilen beziehungsweise 3 Spalten. \\
Die Tupel (1, 2, 3),(4, 5, 6),(7, 8, 9) beschreiben die Zeilen oder Spalten der jeweiligen Bänder. \\
Mit dem Wissen über die Bänder können wir die Symmetrien eines Sudokus formulieren~\cite{russell2006mathematics}: \\
\begin{itemize}
    \item Permutation der 9 Ziffern
    \item Permutation der 3 horizontalen Bänder
    \item Permutation der 3 vertikalen Bänder
    \item Permutation der Zeilen innerhalb eines horizontalen Bandes
    \item Permutation der Spalten innerhalb eines vertikalen Bandes
    \item Spiegelung und Rotation
\end{itemize}
Insgesamt gibt es in etwa 6,7 Trilliarden verschiedene valide Sudokus \cite{felgenhauer2006mathematics}. Unter Berücksichtigung der Symmetrien
gibt es ungefähr 5,5 Milliarden verschiedene valide Sudokus\cite{russell2006mathematics}. \\


\subsection{Unterquadrate}

\begin{figure}[H]
    \centering
    \begin{minipage}{0.48\textwidth}
        \begin{tikzpicture}
            \begin{tikzpicture}
                % Zellgröße
                \def\s{0.7cm}

                % Sudoku-Lösung (kann beliebig angepasst werden)
                \def\sudoku{
                        {1, 2, 3, 4, 5, 6, 7, 8, 9},
                        {6, 5, 9, 1, 7, 8, 2, 3, 4},
                        {4, 7, 8, 3, 9, 2, 5, 6, 1},
                        {2, 9, 7, 6, 1, 3, 8, 4, 5},
                        {8, 4, 6, 9, 2, 5, 3, 1, 7},
                        {3, 1, 5, 8, 4, 7, 9, 2, 6},
                        {9, 6, 2, 7, 8, 4, 1, 5, 3},
                        {7, 8, 4, 5, 3, 1, 6, 9, 2},
                        {5, 3, 1, 2, 6, 9, 4, 7, 8},
                }

                \fill[green!30] (5*\s, 5*\s) rectangle (6*\s, 6*\s);
                \fill[green!30] (4*\s, 5*\s) rectangle (5*\s, 6*\s);

                \fill[green!30] (5*\s, 1*\s) rectangle (6*\s, 2*\s);
                \fill[green!30] (4*\s, 1*\s) rectangle (5*\s, 2*\s);

                % Rasterlinien
                \foreach \x in {0,1,...,9} {
                    \draw[thin] (\x*\s, 0) -- (\x*\s, 9*\s);
                    \draw[thin] (0, \x*\s) -- (9*\s, \x*\s);
                }
                \foreach \x in {0,3,6,9} {
                    \draw[very thick] (\x*\s, 0) -- (\x*\s, 9*\s);
                    \draw[very thick] (0, \x*\s) -- (9*\s, \x*\s);
                }




                % Zahlen eintragen
                \foreach \row [count=\i from 0] in \sudoku {
                    \foreach \num [count=\j from 0] in \row {
                    % Y-Koordinate: Startet oben bei 8.5 und geht pro Zeile 1 runter
                        \node at (\j*\s + 0.5*\s, 8.5*\s - \i*\s) {\num};
                    }
                }
            \end{tikzpicture}
        \end{tikzpicture}
    \end{minipage}
    \hfill
    \begin{minipage}{0.48\textwidth}
        \begin{tikzpicture}
            \begin{tikzpicture}
                % Zellgröße
                \def\s{0.7cm}

                % Sudoku-Lösung (kann beliebig angepasst werden)
                \def\sudoku{
                        {1, 2, 3, 4, 5, 6, 7, 8, 9},
                        {6, 5, 9, 1, 7, 8, 2, 3, 4},
                        {4, 7, 8, 3, 9, 2, 5, 6, 1},
                        {2, 9, 7, 6, 3, 1, 8, 4, 5},
                        {8, 4, 6, 9, 2, 5, 3, 1, 7},
                        {3, 1, 5, 8, 4, 7, 9, 2, 6},
                        {9, 6, 2, 7, 8, 4, 1, 5, 3},
                        {7, 8, 4, 5, 1, 3, 6, 9, 2},
                        {5, 3, 1, 2, 6, 9, 4, 7, 8},
                }

                \fill[green!30] (5*\s, 5*\s) rectangle (6*\s, 6*\s);
                \fill[green!30] (4*\s, 5*\s) rectangle (5*\s, 6*\s);

                \fill[green!30] (5*\s, 1*\s) rectangle (6*\s, 2*\s);
                \fill[green!30] (4*\s, 1*\s) rectangle (5*\s, 2*\s);

                % Rasterlinien
                \foreach \x in {0,1,...,9} {
                    \draw[thin] (\x*\s, 0) -- (\x*\s, 9*\s);
                    \draw[thin] (0, \x*\s) -- (9*\s, \x*\s);
                }
                \foreach \x in {0,3,6,9} {
                    \draw[very thick] (\x*\s, 0) -- (\x*\s, 9*\s);
                    \draw[very thick] (0, \x*\s) -- (9*\s, \x*\s);
                }




                % Zahlen eintragen
                \foreach \row [count=\i from 0] in \sudoku {
                    \foreach \num [count=\j from 0] in \row {
                    % Y-Koordinate: Startet oben bei 8.5 und geht pro Zeile 1 runter
                        \node at (\j*\s + 0.5*\s, 8.5*\s - \i*\s) {\num};
                    }
                }
            \end{tikzpicture}
        \end{tikzpicture}
    \end{minipage}
    \caption{Gemeinsame Beschriftung für beide Sudoku-Gitter}
    \label{fig:gemeinsames_sudoku}
\end{figure}

Innerhalb von vollständig ausgefüllten Sudokus kann es Unterquadrate geben.
Ein Beispiel für ein solches Unterquadrat ist in \cref{fig:gemeinsames_sudoku} gegeben.
Ein Unterquadrat besteht aus vier Zellen in denen zwei Ziffern jeweils doppelt vorkommen.
Je zwei Zellen sind dabei in der gleichen Spalte beziehungsweise in der gleichen Zeile.
Entweder beide Zeilen oder beide Spalten sind im gleichen Band.
Durch die Anordnung der Zellen ist es möglich, ein weiteres gültiges Sudoku zu generieren, nur durch Permutieren der beiden Ziffern im Unterquadrat.
Das heißt, um ein eindeutig lösbares partielles Sudoku aus diesem voll ausgefüllten Sudoku zu erhalten,
muss mindestens eine der Zellen des Unterquadrats Teil des partiellen Sudokus sein.
Einige vollständige Sudokus haben sehr viele Unterquadrate und sind dementsprechend weniger gut geeignet,
um daraus ein partielles Sudoku zu generieren.
Um unsere Liste an vollständigen Sudokus für die Suche zu verbessern, generieren wir nur Sudokus, die keine Unterquadrate enthalten.

\section{Erzeugung der partiellen Sudokus}

\input{Sections/Erzeugung Lösung}


\subsection{Algorithmen}
\subsubsection{Generierungsalgorithmen}


\subsubsection{Lösungsalgorithmen}

TODO in die Überschriften besser einteilen.

Um zu überprüfen, ob bestimmte partielle Sudokus eindeutig lösbar ist, haben wir uns verschieden Lösungsmöglichkeiten überlegt.
Die zwei wichtigsten Überlegungen waren einmal ein CSP-Solver (Constraint Satisfaction Problem) oder ein SAT-Solver (Satisfiability).
Jeweils Ansätze je aus dem Bereich der künstlichen Intelligenz oder aus der theoretischen Informatik.
Um herauszufinden, welche dieser Ansätze eine schnellere Lösung bietet haben wir ein python Skript geschrieben welches diesen Sachverhalt untersuchen sollte~\cite{TODO}.
\begin{enumerate}
    \item CSP-Solver \\
    Um das Problem eines Sudokus in ein CSP zu übertragen haben wir schlicht die Domänen der Zellen auf die Werte $1, \dots, 9$ gesetzt und die Sudoku Regeln jeweils in Constraints übersetzt und anschließend die Domänen der schon vorgegebenen Zellen nur auf die eingegebenen Zahlen eingeschränkt.
    Die Constraints waren also für jede Zelle $z_{i, k}$ einer Reihe, dass $z_{i, k} \neq z_{j, k}$ mit $j \neq i$.
    Das Gleiche gilt für die Spalten, also $z_{i, k} \neq z_{i, l}$ mit $k \neq l$.
    Für die Unterquadrate wurden jeweils die auch wieder geschaut, dass jede Zelle nicht den gleichen Wert haben darf wie eine andere im Unterquadrat.
    \item SAT-Solver \\
    Da der Sat-Solver nur für Variablen mit den Werten wahr oder falsch arbeiten kann.
    Dafür wurde zuerst jeder Zelle mit jeden Wert eine bestimmte Variable zugewiesen.
    \begin{verbatim}
        varnum(r, c, d) = size * size * (r - 1) + size * (c - 1) + d
    \end{verbatim}
    TODO Mathematisches Aufschreiben
    Ein SAT-Solver nimmt Clauses in konjunktiver Normalform entgegen.
    Man musste also für jede Zelle bestimmen, dass mindestens eine Variable wahr ist.
    Außerdem darf maximal eine Variable für jeden einzelnen Wert pro Zelle wahr sein.
    Das musste auch für jede Spalte, jede Reihe und jedes Unterquadrat festgelegt werden.
    Um die Hinweise hinzuzufügen, nimmt man also nun an, dass bestimmt Variablen wahr sind und löst dann mit dieser Annahme das SAT-Problem.
\end{enumerate}

Um die Lösungsansätze miteinander vergleichen zu können, haben wir jeweils 100 4x4, 6x6 und 9x9 sudokus lösen lassen.
Hierfür haben wir einmal den cadical SAT-Solver~\cite{TODO} verwendet und dann noch die verschieden CSP-Solver von TODO~\cite{TODO}.
Wir haben hierfür einmal den durchschnitt der Zeiten pro Solver für das reine lösen ausgewertet und die Anzahl der abgelaufenen Durchläufe.

TODO add graphs
TODO die graphen noch genauer beschreiben

Wie man unschwer erkennen kann, ist der SAT-Solver mit Abstand am schnellsten und löst die Sudokus innerhalb von meist nicht messbarer Zeit.
Auch wenn sich manche CSP-Lösungsverfahren gut gehalten haben, macht es keinen Sinn etwas anderes als den getesteten SAT-Solver zu verwenden.
Wir können mit ähnlicher Performanz rechnen, da auch cadical-rs so wie die Implementierung in Python auf dem gleichen Code in C/C++ laufen.

\subsection{Implementierung}
\subsection{Kommunikation per POST-Anfrage}

Die Kommunikation zwischen Frontend und Backend erfolgt über POST-Anfragen. Das Frontend sendet dabei Daten im JSON-Format an einen definierten API-Endpunkt des Backends.

\subsection{Frontend-Seite}

Das Frontend erstellt ein JSON-Objekt mit den benötigten Daten und schickt es per POST an das Backend. Beispiel mit JavaScript:

\begin{verbatim}
fetch('/api/endpoint', {
  method: 'POST',
  headers: { 'Content-Type': 'application/json' },
  body: JSON.stringify({ key: 'value' })
})
.then(response => response.json())
.then(data => {
  // Daten verarbeiten und UI aktualisieren
});
\end{verbatim}

\subsection{Backend-Seite}

Das Backend empfängt die POST-Anfrage, liest die JSON-Daten aus, verarbeitet sie (z.B. Validierung, Datenbankzugriff) und sendet eine JSON-Antwort zurück.

\subsection{Antwort}

Die Antwort enthält den Status der Verarbeitung und ggf. weitere Daten im JSON-Format. Das Frontend wertet diese aus und reagiert entsprechend (z.B. Anzeige von Erfolg oder Fehler).

\subsection{Zusammenfassung}

Die Kommunikation basiert auf dem Senden und Empfangen von JSON-Daten per POST-Anfrage. Das Frontend baut die Anfrage, das Backend verarbeitet sie und antwortet mit JSON. So findet der Datenaustausch konkret statt.

\subsubsection{Tech-Stack}

Das Frontend ist eine Webanwendung, die vollständig im Browser lauffähig ist.
Diese bietet zwei grundlegende Betriebsmodi für die Erstellung von Sudokus an.
Einmal den Markiermodus, in dem Felder visuell hervorgehoben werden können,
und den Eingabemodus, der eine manuelle Eingabe von Zahlen erlaubt.
Die Anwendung unterstützt dabei Sudoku-Größen von 4x4, 6x6 und 9x9.
\\
Die HTML-Datei enthält die grundlegenden Interface-Komponenten der Anwendung. Hierzu gehören:
\begin{itemize}
    \item Schaltflächen zur Auswahl der Sudoku-Größe (4x4, 6x6, 9x9), mit denen dynamisch ein passendes Raster erzeugt wird.
    \item Umschalter zwischen Markier- und Eingabemodus, die das Verhalten der Zellen beim Anklicken verändern.
    \item Das Grid welches beim anklicken verändert werden kann und wo dann bestimmte Muster oder Formen markiert werden können.
    \item Ein Download-Button, der es ermöglicht, das Sudoku als PDF-Datei zu exportieren.
\end{itemize}

Je nach gewählter Sudoku-Größe wird ein entsprechendes Raster dynamisch erzeugt.
Die Zellen erhalten je nach Modus unterschiedliche Eventlistener:
im Markiermodus lassen sie sich an- und abwählen, im Eingabemodus können Zahlen eingegeben werden.
Die Anwendung erlaubt es, zwischen einem Markier- und einem Eingabemodus zu wechseln.
Der aktuelle Modus wird zentral verwaltet, sodass alle Eingabefunktionen sowie Prüfmechanismen entsprechend angepasst sind.
Nach jeder Benutzereingabe wird automatisch überprüft, ob das aktuelle Sudoku gültig ist.
Im Eingabemodus wird dabei auf doppelte Zahlen in Zeilen, Spalten und Blöcken geachtet.
Im Markiermodus wird geprüft, ob genügend Felder markiert sind und ob die Verteilung über das Sudoku-Feld ausreichend ist (mindestens zwei Zeilen und Spalten pro Block müssen belegt sein).
Fehlerhafte Zellen werden visuell hervorgehoben. Die Benutzereingaben werden in ein JSON-Objekt umgewandelt und an eine serverseitige API übermittelt.
Diese verarbeitet das unvollständige Sudoku, berechnet eine mögliche Lösung und sendet sie zurück an den Client.
Die Lösung wird anschließend im dafür vorgesehenen Bereich angezeigt.
\\
Ein besonderes Augenmerk lag bei der Umsetzung auf der Validierung der Sudoku-Regeln.
Dabei musste berücksichtigt werden, dass bei Sudokus der Größen 4x4 und 6x6 die Blockgrößen variieren (2x2 bzw. 2x3).
Die Logik erkennt automatisch die korrekten Blockdimensionen und wendet die Regeln entsprechend an.
Zudem wurde eine eigene Fehlercodierung entwickelt,
um unterschiedliche Arten von Regelverstößen klar voneinander unterscheiden und gezielt anzeigen zu können.
Dazu gehören doppelte Werte, fehlende Verteilung sowie unvollständige Eingaben.
\\
Eine der größten Herausforderungen war die dynamische Erstellung und Validierung unterschiedlich großer Sudoku-Raster sowie die Anpassung der Prüf- und Eingabelogik an zwei Betriebsmodi.
Ebenso anspruchsvoll war die Entwicklung einer robusten Fehlererkennung,
die bei teilweise ausgefüllten Rätseln dennoch konsistente Rückmeldungen geben kann.
\subsubsection{Backend}
\subsubsection{Interface}

\section{Ergebnisse}

\section{Diskussion}

\section{Fazit und Ausblick}

\bibliographystyle{plain}
\bibliography{references}

\end{document}
